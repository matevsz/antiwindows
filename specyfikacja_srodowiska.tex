\section{Specyfikacja środowiska zadaniowego}
\subsection{Definicja zadań agenta centralnego}\label{ssec:zadaniaCentralny}
%Do głównych zadań agenta należy: Ponadto agent ma za zadanie
Głównym zadaniem agenta centralnego jest odnalezienie wśród wszystkich klientów
i pracowników (wszystkich placówek firmy) tych, którzy związani są z więcej niż jedną placówką.
Dla tych osób agent centralny planuje czas, biorąc pod uwagę następujące ograniczenia:
\begin{enumerate}
	\item{dyspozycyjność i preferencje czasowe tych osób (patrz \ref{sssec:personel} i \ref{sssec:klienci})}
	\item{czas potrzebny na przemieszczenie się pomiędzy placówkami
		i dostępność pomieszczeń o~odpowiednim stałym wyposażeniu (patrz \ref{sssec:miejsca})}
	\item{dostępność przenośnego sprzętu (patrz \ref{sssec:sprzet})}
	\item{uprawnienia personelu (patrz \ref{sssec:personel})}
	\item{zbiór usług przysługujących indywidualnie każdemu klientowi (patrz \ref{sssec:klienci})}
	\item{szczególne wymagania planowanych usług (patrz \ref{sssec:uslugi})}
\end{enumerate}

\subsection{Definicja zadań agentów lokalnych}\label{ssec:zadaniaLokalny}
Agenci lokalni rozpoczynają planowanie czasu dopiero po zakończeniu planowania przez
agenta centralnego i robią to tylko dla osób związanych tylko z przypodządkowaną im placówką.
W~procesie planowania muszą być uwzględnione następujące ograniczenia:
\begin{enumerate}
	\item{dostępność pomieszczeń o odpowiednim stałym wyposażeniu (patrz \ref{sssec:miejsca})}
	\item{dostępność przenośnego sprzętu (patrz \ref{sssec:sprzet})}
	\item{uprawnienia, dyspozycyjność i preferencje personelu (patrz \ref{sssec:personel})}
	\item{zbiór usług przysługujących, dyspozycyjność
		i preferencje klientów (patrz \ref{sssec:klienci})}
	\item{szczególne wymagania planowanych usług (patrz \ref{sssec:uslugi})}
\end{enumerate}


\subsection{Określenie miary wyników działania}
%Podstawowym kryterium oceny sukcesu agenta jest Dodatkowe istotne kryteria to 
Podstawowym kryterium oceny sukcesu agentów jest zaplanowanie wszystkim klientom
wszystkich przysługujących im usług w sposób umożliwiający realne wykonanie planu
dysponując zadanymi zasobami (personelem, miejscem, sprzętem).
Jeżeli nie istnieje taki plan, użytkownik powinien zostać poinformowany o przyczynie,
którą może być zbyt mała liczba pracowników personelu, lub zbyt mała liczba miejsc
do świadczenia usług, lub zbyt mała ilość przenośnego sprzętu.

Dodatkowym kryterium oceny efektu pracy agenta jest jakość planu. Za idealny plan
uważa się taki, który skutecznie bierze pod uwagę \emph{wszystkie} zadane przez
klientów i pracowników preferencje w ramach ich dyspozycyjności, oraz w którym
dla każdej osoby (klienta i pracownika) jednego dnia nie występują przerwy
(potocznie "{okienka}") większe, niż to konieczne\footnote{konieczne dłuższe przerwy
występują głównie w sytuacji, w której klient lub pracownik muszą przemieścić się
pomiędzy oddalonymi od siebie miejscami}.

\subsection{Istotne składowe środowiska}
%Projektowany typ agenta ma działać w środowisku Istotne znaczenie  dla jego funkcjonowania mają następujące aspekty środowiska:  

\subsubsection{Usługi}\label{sssec:uslugi}
Każda ze świadczonych przez firmę usług charakteryzuje się:
\begin{itemize}
	\item{uprawnieniami wymaganymi od personelu}
	\item{wymaganym wyposażeniem miejsca, w którym jest świadczona}
	\item{wymaganym dodatkowym przenośnym sprzętem}
	\item{ilością osób, które mogą odbierać ją jednocześnie}
\end{itemize}
Ten sam rodzaj usługi może przysługiwać różnym klientom w różnej ilości (czasie trwania), ilość jest zadawana tygodniowo, przy czym istnieje ilość domyślna.
Tygodniowy czas trwania usługi może być dzielony na sesje. Domyślnym podziałem tygodniowego czasu \( T_w \) na \( x \) sesji jest podział "po równo"\ \( T_w/x \), przy czym konieczne jest umożliwienie ręcznej ingerencji w ten podział.
Dwie sesje jednej usługi dla jednego klienta domyślnie nie mogą być planowane jednego dnia, ale wskazane jest umożliwienie takiego zachowania agenta w szczególnych przypadkach.
Wszystkie sesje jednej usługi dla jednego klienta domyślnie prowadzone są przez tego samego pracownika personelu. Wskazane jest umożliwienie zaplanowania różnych pracowników dla różnych sesji tej samej usługi dla jednego klienta w szczególnych przypadkach.

\subsubsection{Klienci}\label{sssec:klienci}
Każdy z klientów charakteryzuje się:
\begin{itemize}
	\item{czasem, w którym jest gotowy odbierać usługi (dyspozycyjność)}
	\item{preferencjami dotyczącymi czasu odbierania usług w ramach swojej dyspozycyjności}
	\item{zestawem (zbiorem) przysługujących mu usług}
	\item{przynależnością do 0 lub więcej grup}
\end{itemize}
Gdy klientowi przysługuje usługa, która jest świadczona w okresie dłuższym, niż okres harmonogramowania\footnote{np. kurs programowania trwający 4 semestry, gdy planowany jest 1 semestr}, należy umożliwić wymuszenie przyporządkowania temu klientowi do świadczenia tej usługi konkretnego pracownika personelu. 
Grupie klientów przysługują usługi tak, jakby ta grupa była jednym klientem. Planując świadczenie usług grupowych agent powinien brać pod uwagę dyspozycyjność i preferencje wszystkich klientów należących do grupy.

\subsubsection{Personel}\label{sssec:personel}
Każdy z pracowników personelu charakteryzuje się:
\begin{itemize}
	\item{uprawnieniami do świadczenia konkretnych rodzajów usług}
	\item{dyspozycyjnością, tzn. o jakich porach dnia/tygodnia jest gotów świadczyć usługi}
	\item{preferencjami dotyczącymi czasu w ramach swojej dyspozycyjności}
\end{itemize}
Pracownik personelu może być przyporządkowany konkretnemu klientowi do świadczenia konkretnej usługi, gdy np. okres harmonogramowania jest krótszy niż okres świadczenia danej usługi danemu klientowi (patrz \ref{sssec:klienci}).
Gdy pracownik personelu jednorazowo nie może prowadzić sesji jakiejś usługi, musi istnieć możliwość automatycznego zaplanowania terminu odrobienia danej sesji w innym terminie, z uwzględnieniem dyspozycyjności i preferencji poszkodowanych klientów i danego pracownika.

\subsubsection{Miejsca}\label{sssec:miejsca}
Firma dysponuje miejscami (np. salami), w których świadczone mogą być usługi. Cechy, które posiadają miejsca, a które muszą zostać uwzględnione przy harmonogramowaniu:
\begin{itemize}
	\item{pojemność, tzn. ile osób jednocześnie może korzystać z usługi w danym miejscu}
	\item{wyposażenie, tzn. jaki sprzęt znajduje się na stałe w danym miejscu}
	\item{dostępność, tzn. o jakich porach dnia/tygodnia, jednorazowo lub okresowo, miejsce jest dostępne (nie jest wyłączone z użytkowania)}
\end{itemize}
Miejsca mogą być od siebie odległe na tyle, że konieczne jest uwzględnienie czasu potrzebnego do przemieszczania się pomiędzy nimi.
Gdy miejsce zostaje wyłączone z użytkowania (w skutek np. zdarzenia losowego) musi być możliwość automatycznego harmonogramowania usług w innych miejscach.

\subsubsection{Sprzęt}\label{sssec:sprzet}
Firma dysponuje ruchomym sprzętem, który charakteryzuje się:
\begin{itemize}
	\item{typem}
	\item{określoną ilością danego typu}
\end{itemize}
Każdy egzemplarz (określonego typu, wliczony w ilość danego typu) może być używany w celu świadczenia jakiejś usługi każdorazowo lub jednorazowo.
Sprzęt nie jest stałym wyposażeniem konkretnego miejsca i może być przenoszony.

\subsection{Dostępne/potrzebne efektory}
%Do realizacji postawionych przed agentem zadań potrzebuje on 
Ponieważ wynikiem działania agenta jest harmonogram, plan zagospodarowania czasu klientów, pracowników, miejsc i sprzętu, efektorem potrzebnym do prawidłowego działania jest funkcja utrwalająca wynik pracy agenta. Za najbardziej pożądany sposób utrwalenia uznaje się zapis na dysk.

\subsection{Dostępne/potrzebne sensory}
%Do właściwego postrzegania środowiska agent musi być wyposażony w następujące sensory: 
Ponieważ środowisko, w jakim pracuje agent, jest czysto wirtualne, do właściwego wykonania swojego zadania potrzebuje on być w stanie odczytać warunki początkowe zadania. Na warunki początkowe zadania składają się elementy wymienione w punkcie \ref{ssec:zadania}.

