\section{Specyfikacja środowiska zadaniowego}
%\emph{Wydaje mi się, że powinnimy rozdzielić odpowiedzialnosci dla agentów.}
%
%\emph{Agent centralny}
%\begin{enumerate}
%	\item{dyspozycyjność i preferencje czasowe tych osób}
%	\item{czas potrzebny na przemieszczenie się pomiędzy placówkami}
%	\item{uprawnienia, dyspozycyjność i preferencje personelu}
%	\item{zbiór usług przysługujących indywidualnie każdemu klientowi
%		(albo przypisanie klienta do placówki w danym czasie)}
%	\item{szczególne wymagania planowanych usług
%		(chodzi o ramy czasowe)}
%\end{enumerate}
%
%\emph{Agent lokalny}
%\begin{enumerate}
%	\item{dostępność pomieszczeń o odpowiednim stałym wyposażeniu}
%	\item{dostępność przenośnego sprzętu}
%	\item{szczególne wymagania planowanych usług}
%\end{enumerate}

\subsection{Definicja zadań agenta centralnego}\label{ssec:zadaniaCentralny}
%Do głównych zadań agenta należy: Ponadto agent ma za zadanie
Głównym zadaniem agenta centralnego jest odnalezienie wśród wszystkich klientów
i pracowników (wszystkich placówek firmy) tych, którzy związani są z więcej niż jedną placówką.
Dla tych osób agent centralny planuje czas, biorąc pod uwagę następujące ograniczenia:
\begin{enumerate}
	\item{dyspozycyjność i preferencje czasowe tych osób (patrz \ref{sssec:personel} i \ref{sssec:klienci})}
	\item{czas potrzebny na przemieszczenie się pomiędzy placówkami	(patrz \ref{sssec:miejsca})}
	\item{uprawnienia personelu (patrz \ref{sssec:personel})}
	\item{zbiór usług przysługujących indywidualnie każdemu klientowi (patrz \ref{sssec:klienci})}
	\item{szczególne wymagania planowanych usług (patrz \ref{sssec:uslugi})}
\end{enumerate}
Agent centralny planując czas kojarzy ze sobą odpowiedniego klienta(-ów) i pracownika w celu
realizacji odpowiedniej usługi i przesyła agentowi lokalnemu w odpowiedniej placówce propozycję
takiego skojarzenia (spotkania) w celu weryfikacji. Jeżeli dane skojarzenie jest nierealizowalne,
agent centralny oczekuje informacji o powodzie - może nim być brak wolnych pomieszczeń spełniających
wymagania usługi lub przenośnego sprzętu - i na tej podstawie próbuje wypracować lepszą propozycję.

\subsection{Definicja zadań agentów lokalnych}\label{ssec:zadaniaLokalny}
Agenci lokalni weryfikują propozycje agenta centralnego dotyczące czasu osób (klientów
i pracowników), które są związane z więcej niż jedną placówką (patrz \ref{ssec:zadaniaCentralny}).
Propozycje takie zawierają w sobie skojarzenie klienta(-ów), pracownika, usługi i czasu,
a weryfikacja polega na sprawdzeniu, czy w placówce jest w tym czasie wolne pomieszczenie i sprzęt,
które spełniają wymagania danej usługi. Jeżeli nie, agent lokalny informuje o tym agenta centralnego,
który próbuje wypracować lepszą propozycję.
Agenci lokalni rozpoczynają planowanie czasu dopiero po zakończeniu planowania przez
agenta centralnego i robią to tylko dla osób związanych tylko z przypodządkowaną im placówką.
W procesie planowania muszą być uwzględnione następujące ograniczenia:
\begin{enumerate}
	\item{dyspozycyjność i preferencje czasowe tych osób (patrz \ref{sssec:personel} i \ref{sssec:klienci})}
	\item{uprawnienia personelu (patrz \ref{sssec:personel})}
	\item{zbiór usług przysługujących indywidualnie każdemu klientowi (patrz \ref{sssec:klienci})}
	\item{dostępność pomieszczeń o odpowiednim stałym wyposażeniu (patrz \ref{sssec:miejsca})}
	\item{dostępność przenośnego sprzętu (patrz \ref{sssec:sprzet})}
	\item{szczególne wymagania planowanych usług (patrz \ref{sssec:uslugi})}
\end{enumerate}

%\emph{Uważam, że miary wyników działania agentów też są różne.
%	Agent centralny oczywicie mocno zależy od agentów lokalnych}
%
%\emph{Agent centralny}\\
%\emph{Miarą oceny jest sposób harmonowania zajęć, kursów. Ponieważ domeną agenta są zasoby ludzkie,
%	klienci, personel oraz czas potrzebny do przemieszczania się między placówkami, oceniany być powinien
%	sposób przydzialania czasu na odpowiednie czynnosci: czas trwania kursu, dojazdu między placówkami,
%	nie przydzielony czas między zajęciami(okienka). W tym ujęciu jakoć agenta proporcjonalna do czasu kursów
%	i dojazdów, i odwrotnie proporcjonalna do nie przydzielonego czasu.}
%
%\emph{Agent lokalny} \\
%\emph{Podstawowym zadaniem jest przydzielanie zasobów udostępnianych przez placówkę do realizacji kursów
%	wynikających z przypisanych klientów i personelu dydaktycznego. Biorąc pod uwagę, że odpowiedzialnosć
%	przydziału klientów i personelu jest domeną agenta centralnego, to miarą oceniającego agenta lokalnego
%	jest sposób planowania przydziału pomieszczeń oraz sprzętu. W tym ujęciu problemu wynikiem idealnym jest
%	sytuacja gdy wszystkie pomieszczenia są w pełni wykorzystane w czasie gdy kursy mogą się odbywać.
%	Wykorzystanie sprzętu jest raczej elementem który umożliwia wykorzystanie sal i w mojej ocenie nie powinien
%	być brany pod uwagę do oceny agenta. Jednakże, brak sprzętu może uniemożliwić prowadzić zajęć, przez co może
%	mieć wpływ na wykorzystanie pomieszczeń i dalej ocenę agenta.}

\subsection{Określenie miary wyników działania}\label{ssec:miaraWynikow}
%Podstawowym kryterium oceny sukcesu agenta jest Dodatkowe istotne kryteria to 
Podstawowym kryterium oceny sukcesu zespołu agentów jest zaplanowanie wszystkim klientom
wszystkich przysługujących im usług w sposób umożliwiający realne wykonanie planu
dysponując zadanymi zasobami (personelem, miejscem, sprzętem).
Jeżeli nie istnieje taki plan, użytkownik powinien zostać poinformowany o przyczynie,
którą może być zbyt mała liczba pracowników personelu, lub zbyt mała liczba miejsc
do świadczenia usług, lub zbyt mała ilość przenośnego sprzętu.

Dodatkowym kryterium oceny efektu pracy agentów jest jakość planu. Ponieważ wpływ agentów na plan
jest różny w zależności od typu agenta, miary oceny również są różne i opisane w dalszej części.

\subsubsection{Agent centralny}\label{sssec:miaraWynikowCentralny}
Agent centralny planuje czas, więc ocena musi być oparta na jakości zagospodarowania czasu osób,
za które on odpowiada. Idealne zagospodarowanie czasu to takie, które bierze pod uwagę \emph{wszystkie}
zadane przez klientów i pracowników preferencje w ramach ich dyspozycyjności, oraz w którym nie
występują przerwy (potocznie "{okienka}") większe, niż to konieczne\footnote{konieczne dłuższe przerwy
występują głównie w sytuacji, w której klient lub pracownik muszą przemieścić się
pomiędzy oddalonymi od siebie miejscami}. Każde naruszenie preferencji i wystąpienie okienek jest
związane z uwzględnieniem w ocenie odpowiedniej kary.

\subsubsection{Agenci lokalni}\label{sssec:miaraWynikowLokalny}
Agent lokalny jest odpowiedzialny za planowanie czasu osobom, które związane są tylko z jego placówką,
zatem efekt jego pracy jest oceniany analogicznie, jak agenta centralnego (patrz \ref{sssec:miaraWynikowCentralny}).
Oprócz tego agent lokalny jest \emph{w całości} odpowiedzialny za gospodarowanie pomieszczeniami i sprzętem,
które znajdują się w jego placówce, dlatego jego ocena uwzględnia również tę kwestię.
Idealne zagospodarowanie pomieszczeń to takie, w którym wszystkie pomieszczenia są w pełni wykorzystane
w czasie, gdy kursy mogą się odbywać.

%Tu mam straszny problem z tym, jak w jasny sposób można ocenić, czy pomieszczenia są wykorzystane "w pełni"?
%Bo to, że zajęte są WSZYSTKIE do którejś godziny jest zbytnim uproszczeniem..


\subsection{Istotne składowe środowiska}
%Projektowany typ agenta ma działać w środowisku Istotne znaczenie  dla jego funkcjonowania mają następujące aspekty środowiska:  

\subsubsection{Usługi}\label{sssec:uslugi}
Każda ze świadczonych przez firmę usług charakteryzuje się:
\begin{itemize}
	\item{uprawnieniami wymaganymi od personelu}
	\item{wymaganym wyposażeniem miejsca, w którym jest świadczona}
	\item{wymaganym dodatkowym przenośnym sprzętem}
	\item{ilością osób, które mogą odbierać ją jednocześnie}
\end{itemize}
Ten sam rodzaj usługi może przysługiwać różnym klientom w różnej ilości (czasie trwania), ilość jest zadawana tygodniowo, przy czym istnieje ilość domyślna.
Tygodniowy czas trwania usługi może być dzielony na sesje. Domyślnym podziałem tygodniowego czasu \( T_w \) na \( x \) sesji jest podział "po równo"\ \( T_w/x \), przy czym konieczne jest umożliwienie ręcznej ingerencji w ten podział.
Dwie sesje jednej usługi dla jednego klienta domyślnie nie mogą być planowane jednego dnia, ale wskazane jest umożliwienie takiego zachowania agenta w szczególnych przypadkach.
Wszystkie sesje jednej usługi dla jednego klienta domyślnie prowadzone są przez tego samego pracownika personelu. Wskazane jest umożliwienie zaplanowania różnych pracowników dla różnych sesji tej samej usługi dla jednego klienta w szczególnych przypadkach.

\subsubsection{Klienci}\label{sssec:klienci}
Każdy z klientów charakteryzuje się:
\begin{itemize}
	\item{czasem, w którym jest gotowy odbierać usługi (dyspozycyjność)}
	\item{preferencjami dotyczącymi czasu odbierania usług w ramach swojej dyspozycyjności}
	\item{zestawem (zbiorem) przysługujących mu usług}
	\item{przynależnością do 0 lub więcej grup}
\end{itemize}
Gdy klientowi przysługuje usługa, która jest świadczona w okresie dłuższym, niż okres harmonogramowania\footnote{np. kurs programowania trwający 4 semestry, gdy planowany jest 1 semestr}, należy umożliwić wymuszenie przyporządkowania temu klientowi do świadczenia tej usługi konkretnego pracownika personelu. 
Grupie klientów przysługują usługi tak, jakby ta grupa była jednym klientem. Planując świadczenie usług grupowych agent powinien brać pod uwagę dyspozycyjność i preferencje wszystkich klientów należących do grupy.

\subsubsection{Personel}\label{sssec:personel}
Każdy z pracowników personelu charakteryzuje się:
\begin{itemize}
	\item{uprawnieniami do świadczenia konkretnych rodzajów usług}
	\item{dyspozycyjnością, tzn. o jakich porach dnia/tygodnia jest gotów świadczyć usługi}
	\item{preferencjami dotyczącymi czasu w ramach swojej dyspozycyjności}
\end{itemize}
Pracownik personelu może być przyporządkowany konkretnemu klientowi do świadczenia konkretnej usługi, gdy np. okres harmonogramowania jest krótszy niż okres świadczenia danej usługi danemu klientowi (patrz \ref{sssec:klienci}).
Gdy pracownik personelu jednorazowo nie może prowadzić sesji jakiejś usługi, musi istnieć możliwość automatycznego zaplanowania terminu odrobienia danej sesji w innym terminie, z uwzględnieniem dyspozycyjności i preferencji poszkodowanych klientów i danego pracownika.

\subsubsection{Miejsca}\label{sssec:miejsca}
Firma dysponuje miejscami (np. salami), w których świadczone mogą być usługi. Cechy, które posiadają miejsca, a które muszą zostać uwzględnione przy harmonogramowaniu:
\begin{itemize}
	\item{pojemność, tzn. ile osób jednocześnie może korzystać z usługi w danym miejscu}
	\item{wyposażenie, tzn. jaki sprzęt znajduje się na stałe w danym miejscu}
	\item{dostępność, tzn. o jakich porach dnia/tygodnia, jednorazowo lub okresowo, miejsce jest dostępne (nie jest wyłączone z użytkowania)}
\end{itemize}
Miejsca mogą być od siebie odległe na tyle, że konieczne jest uwzględnienie czasu potrzebnego do przemieszczania się pomiędzy nimi.
Gdy miejsce zostaje wyłączone z użytkowania (w skutek np. zdarzenia losowego) musi być możliwość automatycznego harmonogramowania usług w innych miejscach.

\subsubsection{Sprzęt}\label{sssec:sprzet}
Firma dysponuje ruchomym sprzętem, który charakteryzuje się:
\begin{itemize}
	\item{typem}
	\item{określoną ilością danego typu}
\end{itemize}
Każdy egzemplarz (określonego typu, wliczony w ilość danego typu) może być używany w celu świadczenia jakiejś usługi każdorazowo lub jednorazowo.
Sprzęt nie jest stałym wyposażeniem konkretnego miejsca i może być przenoszony.


%\emph{nie mam pomysłu jak to opisać (sensory i efektory), ale wydaje mi się że nasz projekt
%ma cechy systemu rozproszonego, w związku z tym powinno co być odnonie dostępu do internetu/LAN/WAN,
%	 dodatkowo komunikacja między klientami.}

\subsection{Dostępne/potrzebne efektory}
%Do realizacji postawionych przed agentem zadań potrzebuje on 
Ponieważ wynikiem działania agenta jest harmonogram, plan zagospodarowania czasu klientów,
pracowników, miejsc i sprzętu, efektorem potrzebnym do prawidłowego działania jest funkcja
utrwalająca wynik pracy agenta. Za najbardziej pożądany sposób utrwalenia uznaje się zapis
na dysk - lokalny, sieciowy, baza danych.

\subsection{Dostępne/potrzebne sensory}
%Do właściwego postrzegania środowiska agent musi być wyposażony w następujące sensory: 
Ponieważ środowisko, w jakim pracuje agent, jest czysto wirtualne, do właściwego wykonania
swojego zadania potrzebuje on być w stanie odczytać warunki początkowe zadania. Na warunki
początkowe zadania składają się elementy wymienione w punkcie \ref{ssec:zadania}, które 
przechowywane są na dysku - lokalnym, sieciowy, w bazie danych.

