\section{Własności środowiska zadaniowego}
\subsection{Obserwowalność a poznawalność}
Środowisko zadaniowe jest w pełni obserwowalne i w pełni poznawalne.Wszystkie zmienne zadania są określone w momencie
 rozpoczęcia pracy agentów i dostępne w bazie danych w każdej chwili.
 
\subsection{Przewidywalność}
Środowisko zadaniowe jest w pełni przewidywalne - deterministyczne. Każda akcja agentów kończy się zawsze zaplanowanym
 wynikiem, nie ma innych czynników wpływających na środowisko.

\subsection{Sekwencyjność}
Środowisko zadaniowe jest sekwencyjne. Jednorazowe zaplanowanie czasu wpływa na dostępność wszystkich czynników,
w związku z czym każda decyzja wpływa na wszystkie następne decyzje.

\subsection{Zmienność}
Środowisko zadaniowe jest statyczne. Poza grupą agentów w środowisku nie działa żaden inny czynnik je zmieniający.
Zadanie jest podzielone między agenta centralnego i lokalnych w taki sposób, że operują na częściach środowiska, na których
nikt inny w tym samym czasie nie operuje.

\subsection{Ciągłość}
We wszystkich aspektach środowisko zadaniowe jest dyskretne. Żadna z wielkości nie jest ciągła. Czas analizowany jest
z rozdzielczością minutową, a więc także jest dyskretny.

\subsection{Modalność sensoryczna}
%Wg mnie agent centralny jest 2-modalny (baza danych i inni agenci)
%Agenci w placówkach są 3-modalni: użytkownik, baza danych i agent centralny.
%Nie jestem pewien więc nie wpisuje

%Nie do końca wiem, czy operowanie na środowisku w postaci bazy danych powoduje, że środowisko jest monomodalne?
%Czy każda `tabela' powinna być traktowana jako jedno ze źródeł informacji?

\subsection{Populacja agentów}
W środowisku zadaniowym działa tylko nasza grupa agentów. Ze względu na podział na agenta centralnego i lokalnych
jest to populacja typu gwiazda.

%W sumie ok