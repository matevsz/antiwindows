\section{Model podsystemu analityczno-decyzyjnego}
\subsection{Określenie modelu agenta}
Do realizacji zadań postawionych w projektowanym systemie konieczne jest przyjęcie modelu agenta Bazującego na przydatności.

\subsection{Baza reguł warunek – akcja}

\subsubsection{Agent centralny}

\begin{tabular}{p{7cm}|p{7cm}}
Warunek & Akcja\\
\hline
Nie istnieje klient należący do więcej niż jednego 1.01, dla którego nie jest puste 1.03 & 1.12\\

Istnieje klient należący do więcej niż jednego 1.01, dla którego nie jest puste 1.03 &
Dla danego klienta wykonaj 1.04, 1.05,
Wybierz z 1.02 pracownika mogącego świadczyć tę usługę i wykonaj dla niego 1.06, 1.07,
Zaproponuj kombinowaną propozycję i wykonaj 1.08 \\

Negatywne 1.08 &
Wskaż inny czas niż ostatnio i wykonaj 1.08
LUB wybierz z 1.02 innego niż ostatnio pracownika mogącego świadczyć tę usługę,
Wykonaj dla niego 1.06, 1.07,
Zaproponuj kombinowaną propozycję i wykonaj 1.08
LUB dla dowolnego z 1.09 wykonaj 1.10 i dla uwolnionego czasu wykonaj 1.08\\



\end{tabular}

\subsubsection{Agent lokalny}

\begin{tabular}{p{7cm}|p{7cm}}
Warunek & Akcja\\
\hline
2.12 & Rozpocznij samodzielne lokalne planowanie\\

Istnieje klient należący do 4.01, dla którego nie jest puste 4.03 &
Dla danego klienta wykonaj 4.04, 4.05,
Wybierz z 4.02 pracownika mogącego świadczyć tę usługę i wykonaj dla niego 4.06, 4.07,
Wybierz z 4.14 pomieszczenie, w którym można świadczyć tę usługę i wykonaj dla niego 4.10,
Wybierz z 4.15 sprzęt, który może służyć do świadczenia tej usługi i wykonaj dla niego 4.11,
Dla odpowiedniej kombinacji wykonaj 4.12\\


\end{tabular}

\subsection{Model środowiska}
\subsection{Podsystem wyznaczania strategii}
\subsection{Podsystem oceny przydatności}
\subsection{Podsystem uczący}
\subsection{Podsystem generacji zadań}
