\documentclass[oneside]{article}
\usepackage[a4paper]{geometry}
\usepackage[polish]{babel}
\usepackage{fancyhdr}
\usepackage[utf8]{inputenc}
\usepackage[T1]{fontenc}
\usepackage{hyperref}
\usepackage{amsmath}
%\usepackage{varioref}

\begin{document}
\begin{titlepage}
\centering
{\Huge Zachodniopomorski Uniwesytet\\Technologiczny w Szczecinie}
\rule{\linewidth}{0.4pt}
\par\vspace{1cm}
{\Large Wydział Informatyki}
\par\vspace{3cm}
{\Huge Projektowanie Inteligentnych\\Systemów Informatycznych}
\par\vspace{0.5cm}{\Large Projekt systemu agentowego}
\par\vspace{2cm}{\Huge\textbf{Antiwindows}}
\par\vspace{5cm}
\raggedright Marcin Michalak\\Mateusz Słomiany
\par\vfill
\centering
{\Large Szczecin 2016}
\end{titlepage}
\section{Cel projektu}
Celem projektu jest opracowanie modelu systemu agentowego
wykonującego zadanie planowania harmonogramu świadczenia usług
z uwzględnieniem dostępności zasobów i preferencji klientów i pracowników.

\section{Zakres systemu agentowego (kontekst)} \label{Zakres}
%System swoim zasięgiem obejmuje
System harmonogramuje świadczenie usług klientom wykorzystując dostępne zasoby firmy, którymi są:
\begin{itemize}
	\item{placówki}
	\item{sprzęt}
	\item{personel}
\end{itemize}
Planowany jest okres jednego tygodnia.

Użytkownikami systemu są:
\begin{itemize}
	\item{planista}
	\item{personel}
	\item{klienci}
\end{itemize}
Planista działa na poziomie całej firmy - wszystkich jej placówek jednocześnie.
System agentowy podzielony jest na część (agenta) centralną i część (agentów)
lokalną. Część centralna odpowiedzialna jest za zaplanowanie czasu wszystkim
osobom (klientom i pracownikom), które związane są z więcej niż jedną placówką.
Część lokalna odpowiedzialna jest za zaplanowanie czasu wszystkim pozostałym
osobom.

\section{Specyfikacja środowiska zadaniowego}
\subsection{Definicja zadań}\label{ssec:zadania}
%Do głównych zadań agenta należy: Ponadto agent ma za zadanie
Głównym zadaniem agenta jest zaplanowanie świadczenia usług klientom w czasie na przestrzeni tygodnia. W procesie planowania każdemu klientowi przysługujących mu usług, przydzielone zostają na poczet ich realizacji konieczne do tego zasoby firmy: pracownicy, miejsca, sprzęt i czas. W procesie planowania muszą być uwzględnione następujące ograniczenia:
\begin{enumerate}
	\item{dostępność pomieszczeń o odpowiednim stałym wyposażeniu (patrz \ref{sssec:miejsca})}
	\item{dostępność przenośnego sprzętu (patrz \ref{sssec:sprzet})}
	\item{uprawnienia, dyspozycyjność i preferencje personelu (patrz \ref{sssec:personel})}
	\item{dyspozycyjność i preferencje klientów (patrz \ref{sssec:klienci})}
	\item{szczególne wymagania planowanych usług (patrz \ref{sssec:uslugi})}
\end{enumerate}

\subsection{Określenie miary wyników działania}
%Podstawowym kryterium oceny sukcesu agenta jest Dodatkowe istotne kryteria to 
Podstawowym kryterium oceny sukcesu agenta jest zaplanowanie wszystkim klientom wszystkich przysługujących im usług w sposób umożliwiający realne wykonanie planu dysponując zadanymi zasobami (personelem, miejscem, sprzętem).
Jeżeli nie istnieje taki plan, użytkownik powinien zostać poinformowany o przyczynie, którą może być zbyt mała liczba pracowników personelu, lub zbyt mała liczba miejsc do świadczenia usług, lub zbyt mała ilość przenośnego sprzętu.

Dodatkowym kryterium oceny efektu pracy agenta jest jakość planu. Za idealny plan uważa się taki, który skutecznie bierze pod uwagę \emph{wszystkie} zadane przez klientów i pracowników preferencje w ramach ich dyspozycyjności,
oraz w którym dla każdej osoby (klienta i pracownika) jednego dnia nie występują przerwy (potocznie "okienka") większe, niż to konieczne\footnote{konieczne dłuższe przerwy występują głównie w sytuacji, w której klient lub pracownik muszą przemieścić się pomiędzy oddalonymi od siebie miejscami}.

\subsection{Istotne składowe środowiska}
%Projektowany typ agenta ma działać w środowisku Istotne znaczenie  dla jego funkcjonowania mają następujące aspekty środowiska:  

\subsubsection{Usługi}\label{sssec:uslugi}
Każda ze świadczonych przez firmę usług charakteryzuje się:
\begin{itemize}
	\item{uprawnieniami wymaganymi od personelu}
	\item{wymaganym wyposażeniem miejsca, w którym jest świadczona}
	\item{wymaganym dodatkowym przenośnym sprzętem}
	\item{ilością osób, które mogą odbierać ją jednocześnie}
\end{itemize}
Ten sam rodzaj usługi może przysługiwać różnym klientom w różnej ilości (czasie trwania), ilość jest zadawana tygodniowo, przy czym istnieje ilość domyślna.
Tygodniowy czas trwania usługi może być dzielony na sesje. Domyślnym podziałem tygodniowego czasu \( T_w \) na \( x \) sesji jest podział "po równo"\ \( T_w/x \), przy czym konieczne jest umożliwienie ręcznej ingerencji w ten podział.
Dwie sesje jednej usługi dla jednego klienta domyślnie nie mogą być planowane jednego dnia, ale wskazane jest umożliwienie takiego zachowania agenta w szczególnych przypadkach.
Wszystkie sesje jednej usługi dla jednego klienta domyślnie prowadzone są przez tego samego pracownika personelu. Wskazane jest umożliwienie zaplanowania różnych pracowników dla różnych sesji tej samej usługi dla jednego klienta w szczególnych przypadkach.

\subsubsection{Klienci}\label{sssec:klienci}
Każdy z klientów charakteryzuje się:
\begin{itemize}
	\item{czasem, w którym jest gotowy odbierać usługi (dyspozycyjność)}
	\item{preferencjami dotyczącymi czasu odbierania usług w ramach swojej dyspozycyjności}
	\item{zestawem (zbiorem) przysługujących mu usług}
	\item{przynależnością do 0 lub więcej grup}
\end{itemize}
Gdy klientowi przysługuje usługa, która jest świadczona w okresie dłuższym, niż okres harmonogramowania\footnote{np. kurs programowania trwający 4 semestry, gdy planowany jest 1 semestr}, należy umożliwić wymuszenie przyporządkowania temu klientowi do świadczenia tej usługi konkretnego pracownika personelu. 
Grupie klientów przysługują usługi tak, jakby ta grupa była jednym klientem. Planując świadczenie usług grupowych agent powinien brać pod uwagę dyspozycyjność i preferencje wszystkich klientów należących do grupy.

\subsubsection{Personel}\label{sssec:personel}
Każdy z pracowników personelu charakteryzuje się:
\begin{itemize}
	\item{uprawnieniami do świadczenia konkretnych rodzajów usług}
	\item{dyspozycyjnością, tzn. o jakich porach dnia/tygodnia jest gotów świadczyć usługi}
	\item{preferencjami dotyczącymi czasu w ramach swojej dyspozycyjności}
\end{itemize}
Pracownik personelu może być przyporządkowany konkretnemu klientowi do świadczenia konkretnej usługi, gdy np. okres harmonogramowania jest krótszy niż okres świadczenia danej usługi danemu klientowi (patrz \ref{sssec:klienci}).
Gdy pracownik personelu jednorazowo nie może prowadzić sesji jakiejś usługi, musi istnieć możliwość automatycznego zaplanowania terminu odrobienia danej sesji w innym terminie, z uwzględnieniem dyspozycyjności i preferencji poszkodowanych klientów i danego pracownika.

\subsubsection{Miejsca}\label{sssec:miejsca}
Firma dysponuje miejscami (np. salami), w których świadczone mogą być usługi. Cechy, które posiadają miejsca, a które muszą zostać uwzględnione przy harmonogramowaniu:
\begin{itemize}
	\item{pojemność, tzn. ile osób jednocześnie może korzystać z usługi w danym miejscu}
	\item{wyposażenie, tzn. jaki sprzęt znajduje się na stałe w danym miejscu}
	\item{dostępność, tzn. o jakich porach dnia/tygodnia, jednorazowo lub okresowo, miejsce jest dostępne (nie jest wyłączone z użytkowania)}
\end{itemize}
Miejsca mogą być od siebie odległe na tyle, że konieczne jest uwzględnienie czasu potrzebnego do przemieszczania się pomiędzy nimi.
Gdy miejsce zostaje wyłączone z użytkowania (w skutek np. zdarzenia losowego) musi być możliwość automatycznego harmonogramowania usług w innych miejscach.

\subsubsection{Sprzęt}\label{sssec:sprzet}
Firma dysponuje ruchomym sprzętem, który charakteryzuje się:
\begin{itemize}
	\item{typem}
	\item{określoną ilością danego typu}
\end{itemize}
Każdy egzemplarz (określonego typu, wliczony w ilość danego typu) może być używany w celu świadczenia jakiejś usługi każdorazowo lub jednorazowo.
Sprzęt nie jest stałym wyposażeniem konkretnego miejsca i może być przenoszony.

\subsection{Dostępne/potrzebne efektory}
%Do realizacji postawionych przed agentem zadań potrzebuje on 
Ponieważ wynikiem działania agenta jest harmonogram, plan zagospodarowania czasu klientów, pracowników, miejsc i sprzętu, efektorem potrzebnym do prawidłowego działania jest funkcja utrwalająca wynik pracy agenta. Za najbardziej pożądany sposób utrwalenia uznaje się zapis na dysk.

\subsection{Dostępne/potrzebne sensory}
%Do właściwego postrzegania środowiska agent musi być wyposażony w następujące sensory: 
Ponieważ środowisko, w jakim pracuje agent, jest czysto wirtualne, do właściwego wykonania swojego zadania potrzebuje on być w stanie odczytać warunki początkowe zadania. Na warunki początkowe zadania składają się elementy wymienione w punkcie \ref{ssec:zadania}.


\section{Własności środowiska zadaniowego}
\subsection{Obserwowalność a poznawalność}
\subsection{Przewidywalność}
\subsection{Sekwencyjność}
\subsection{Zmienność}
\subsection{Ciągłość}
\subsection{Modalność sensoryczna}
\subsection{Populacja agentów}
\section{Model podsystemu percepcyjnego}
\subsection{Specyfikacja sensorów (typ, ciągłość pracy)}
\subsection{Specyfikacja postrzeżeń (typ, wielkość, J.M, próg reakcji, zadana wartość, kod postrzeżenia \(P_x\))}
\section{Model podsystemu wykonawczego}
\subsection{Specyfikacja efektorów}

\subsubsection{Agent centralny}

\begin{tabular}{c|p{3cm}|c|p{2cm}|p{4cm}}
LP		& Nazwa			& typ			& ciągłość pracy (częstotliwość) & opis \\
\hline
1		& połączenie z agentem lokalnym	& interface TCP/IP	& dyskretny (na żądanie) & Interfejs do wysyłania informacji do agenta lokalnego lub zapytań odnośnie zajęć klienta w danej placówce \\
\end{tabular}

\subsubsection{Agent lokalny}

\begin{tabular}{c|p{3cm}|c|p{2cm}|p{4cm}}
LP		& Nazwa			& typ		& ciągłość pracy (częstotliwość)	& opis \\
\hline
1		& Strona kliencka		& Strona http	& dyskretny (na żądanie)	& Wyświetlanie informacji o usługach dostępnych w placówce, informacji o zarezerwowanych usługach, wyświetlanie grafików wyświetlanie ważnych informacji (np odwołanie zajęć)\\
2		& połączenie z agentem centralnym		& interface TCP/IP & dyskretny (na żądanie)	& Interfejs do wysyłania informacji do agenta centralnego lub zapytań odnośnie zajęć klienta w innych placówkach \\
3		& Baza danych		& interface SQL	& dyskretny (na żądanie) & Interfejs do utrwalania efektów działania agentów.
\end{tabular}



\subsection{Opis skutków działania}

\subsubsection{Agent centralny}

\begin{tabular}{c|p{5cm}|c}
Nr efektora & Nazwa działania & Kod\\
\hline
1 & Rozesłanie do agentów lokalnych czy mają jako użytkownika wskazanego klienta & 1.01\\
1 & Odesłanie do klienta lokalnego informacji, że wskazany klient jest unikalny & 1.02\\
1 & Wysłanie do agentów lokalnych rządania podania informacji o kliencie & 1.03\\
1 & Zaplanowanie i wysłanie nowego planu klienta & 1.04\\
\end{tabular}

\subsubsection{Agent lokalny}

\begin{tabular}{c|p{5cm}|c}
Nr efektora & Nazwa działania & Kod\\
\hline
1 & Wysłanie strony do autoryzacji iuwierzytelnienia & 1.01\\
1 & Wyswietl stronę użytkownika personel & 1.02\\
1 & Wyswietl stronę użytkownika klient & 1.03\\
1 & Wyswietl stronę z błędem autoryzacji & 1.04\\
1 & Wyswietl informację o usunięciu usługi & 1.05\\
1 & Wyswietl informację o nie usunięciu usługi & 1.06\\
1 & Wyswietl informację o dodaniu usługi & 1.07\\
1 & Wyswietl informację o nie dodaniu usługi & 1.08\\
1 & Wyswietl informację o dodaniu preferencji czasowych & 1.09\\
1 & Wyswietl informację o nie dodaniu preferencji czasowych & 1.10\\
1 & Wyswietl informację o dodaniu dostępnosci użytkownika & 1.11\\
1 & Wyswietl informację o nie dodaniu dostępnosci użytkownika & 1.12\\
1 & Potwierdzenie dodania użytkownika i wyswietlenie strony logowania & 1.13\\
\hline
2 & Wysłanie informacji do agenta centralnego czy wskazany klient jest unikalny & 2.01\\
2 & Wysłanie pełnej informacji o kliencie, oraz usług zaplanowanych i nie zaplanowanych z poptrzebnymi informacjami & 2.02\\
\end{tabular}

\begin{tabular}{c|p{5cm}|c}
Nr efektora & Nazwa działania & Kod\\
\hline
3 & Usunięcie usługi z bazy & 3.01\\
3 & Dodaj preferencje czasowe użytkownika & 3.02\\
3 & Usuń preferencje czasowe użytkownika & 3.03\\
3 & Dodaj dostępnosć użytkownika & 3.04\\
3 & Usuń dostępnosć użytkownika & 3.05\\
3 & Sprawdź czy dany użytkownik istnieje & 3.06\\
3 & Dodaj nowego użytkownika do bazy & 3.07\\
3 & Sprawdź czy dany użytkownik jest aktywny (swiadczy lub korzysta z usług) & 3.08\\
3 & Samodzielne planowanie i zapis danych w bazie & 3.09\\
3 & Usunięcie zajęć użytkownika i zapisanie nowych & 3.10\\
\end{tabular}




\section{Model podsystemu analityczno-decyzyjnego}
\subsection{Określenie modelu agenta}
Do realizacji zadań postawionych w projektowanym systemie konieczne jest przyjęcie modelu agenta …
\subsection{Baza reguł warunek – akcja}
\subsection{Model środowiska}
\subsection{Podsystem wyznaczania strategii}
\subsection{Podsystem oceny przydatności}
\subsection{Podsystem uczący}
\subsection{Podsystem generacji zadań}

\section{Semantyczny model bazy danych środowiska zadaniowego (SERM)}

\end{document}
