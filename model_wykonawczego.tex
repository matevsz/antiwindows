\section{Model podsystemu wykonawczego}
\subsection{Specyfikacja efektorów}

\subsubsection{Agent centralny}

\begin{tabular}{c|p{3cm}|c|p{2cm}|p{4cm}}
LP		& Nazwa			& typ			& ciągłość pracy (częstotliwość) & opis \\
\hline
1		& połączenie z agentem lokalnym	& interface TCP/IP	& dyskretny (na żądanie) & Interfejs do wysyłania informacji do agenta lokalnego lub zapytań odnośnie zajęć klienta w danej placówce \\
\end{tabular}

\subsubsection{Agent lokalny}

\begin{tabular}{c|p{3cm}|c|p{2cm}|p{4cm}}
LP		& Nazwa			& typ		& ciągłość pracy (częstotliwość)	& opis \\
\hline
1		& Strona kliencka		& Strona http	& dyskretny (na żądanie)	& Wyświetlanie informacji o usługach dostępnych w placówce, informacji o zarezerwowanych usługach, wyświetlanie grafików wyświetlanie ważnych informacji (np. odwołanie zajęć)\\
2		& połączenie z agentem centralnym		& interface TCP/IP & dyskretny (na żądanie)	& Interfejs do wysyłania informacji do agenta centralnego lub zapytań odnośnie zajęć klienta w innych placówkach \\
3		& Baza danych		& interface SQL	& dyskretny (na żądanie) & Interfejs do utrwalania efektów działania agentów.
\end{tabular}



\subsection{Opis skutków działania}

\subsubsection{Agent centralny}

\begin{tabular}{c|p{5cm}|c}
Nr efektora & Nazwa działania & Kod\\
\hline
1 & Rozesłanie do agentów lokalnych zapytania czy mają jako użytkownika wskazanego klienta & 1.01\\
1 & Odesłanie do klienta lokalnego informacji, że wskazany klient jest unikalny & 1.02\\
1 & Wysłanie do agentów lokalnych rządania podania informacji o kliencie & 1.03\\
1 & Zaplanowanie i wysłanie nowego planu klienta & 1.04\\
\end{tabular}

\subsubsection{Agent lokalny}

\begin{tabular}{c|p{5cm}|c}
Nr efektora & Nazwa działania & Kod\\
\hline
1 & Wysłanie strony do autoryzacji i uwierzytelnienia & 1.01\\
1 & Wyświetl stronę użytkownika personel & 1.02\\
1 & Wyświetl stronę użytkownika klient & 1.03\\
1 & Wyświetl stronę z błędem autoryzacji & 1.04\\
1 & Wyświetl informację o usunięciu usługi & 1.05\\
1 & Wyświetl informację o nie usunięciu usługi & 1.06\\
1 & Wyświetl informację o dodaniu usługi & 1.07\\
1 & Wyświetl informację o nie dodaniu usługi & 1.08\\
1 & Wyświetl informację o dodaniu preferencji czasowych & 1.09\\
1 & Wyświetl informację o nie dodaniu preferencji czasowych & 1.10\\
1 & Wyświetl informację o dodaniu dostępnosci użytkownika & 1.11\\
1 & Wyświetl informację o nie dodaniu dostępnosci użytkownika & 1.12\\
1 & Potwierdzenie dodania użytkownika i wyświetlenie strony logowania & 1.13\\
\hline
2 & Wysłanie informacji do agenta centralnego czy wskazany klient jest unikalny & 2.01\\
2 & Wysłanie pełnej informacji o kliencie, oraz usług zaplanowanych i nie zaplanowanych z potrzebnymi informacjami & 2.02\\
\end{tabular}

\begin{tabular}{c|p{5cm}|c}
Nr efektora & Nazwa działania & Kod\\
\hline
3 & Usunięcie usługi z bazy & 3.01\\
3 & Dodaj preferencje czasowe użytkownika & 3.02\\
3 & Usuń preferencje czasowe użytkownika & 3.03\\
3 & Dodaj dostępność użytkownika & 3.04\\
3 & Usuń dostępność użytkownika & 3.05\\
3 & Sprawdź, czy dany użytkownik istnieje & 3.06\\
3 & Dodaj nowego użytkownika do bazy & 3.07\\
3 & Sprawdź, czy dany użytkownik jest aktywny (świadczy lub korzysta z usług) & 3.08\\
3 & Samodzielne planowanie i zapis danych w bazie & 3.09\\
3 & Usunięcie zajęć użytkownika i zapisanie nowych & 3.10\\
\end{tabular}



